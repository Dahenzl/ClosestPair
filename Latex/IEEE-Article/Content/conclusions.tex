\section{Conclusiones}
Con la realización de este experimento pudimos determinar la importancia de considerar diferentes técnicas para el diseño de algoritmos que deseemos implementar en nuestros programas, con el fin de escoger el que tenga una mayor eficiencia algorítmica, pues notamos en los resultados del experimento que a pesar de que ambas funciones al final retornan lo mismo, el tiempo de ejecución, el número de iteraciones y la complejidad espacial varían mucho entre ellos, lo que nos permite elegir el que mejor nos convenga. 

Además, pudimos observar que, en la mayoría de los casos, los algoritmos recursivos que siguen la técnica de diseño de “divide y vencerás” son más eficientes tanto temporal como espacialmente que los algoritmos iterativos que usan la técnica de diseño de “fuerza bruta”, pues notamos que en la función recursiva solo se realizan comparaciones que son totalmente necesarias, mientras que en la función iterativa se realizan todas las comparaciones posibles, a pesar de que muchas de estas no sean necesarias y no alteren el resultados.  Esto lo pudimos comprobar en las gráficas realizadas a partir de los resultados del experimento, pues notamos que el algoritmo recursivo tubo una complejidad temporal de $O(N)$ y el iterativo tuvo una complejidad temporal de $O(N^2)$.

Por último, a partir de los resultados de nuestro experimento podemos concluir que los algoritmos que siguen la técnica de diseño de “divide y vencerás” tienen una menor complejidad temporal y espacial, y por ende una mayor eficiencia logarítmica en la solución de ese problema, que los algoritmos iterativos que siguen la técnica de diseño de “fuerza bruta”. 
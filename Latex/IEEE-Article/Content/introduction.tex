\section{Introducción}
\IEEEPARstart{U}{n} algoritmo es un procedimiento para resolver un problema en particular en una serie de pasos dado unos datos de entrada finitos, estos algoritmos se pueden de diseñar de diversas maneras dependiendo de la creatividad y las ideas de la persona que los realice, sin embargo, existen algunas técnicas de diseño que dan una base sobre la cual desarrollarlo enfocando el problema desde diferentes puntos de vista lo que hace que se resuelvan de diferentes maneras, cada una teniendo sus ventajas y desventajas dependiendo cual sea el problema, pues no hay una estrategia que sea mejor que otra en todos los casos. Dos de las técnicas más comunes en el diseño de algoritmos son “divide y vencerás” y “fuerza bruta o búsqueda exhaustiva”. \cite{DesignTechniques} \\

Divide y vencerás parte de un concepto del cual precede su nombre, pues se basa en seguir tres pasos, primero dividir el problema en sub-problemas más pequeños, luego “conquistar” estos problemas llamándolos de manera recursiva hasta que todos se solucionen y por último combinar todos los sub-problemas para conseguir la solución final del problema principal. Esta técnica reduce la complejidad temporal de la solución de los problemas y usa de manera más eficiente la memoria cache sin ocupar demasiado espacio, sin embargo esta eficiencia depende de la manera en la que se implemente la lógica, en algunos casos la recursividad puede ser lenta y el concepto puede ser algo complejo de aplicar para algunos problemas. \cite{DivideAndConquer}\\

Fuerza bruta parte de un concepto más simple e intuitivo de resolución de problemas, en donde todas las soluciones y caminos posibles son considerados para posteriormente encontrar la solución del problema principal, lo que garantiza que se encontrar la solución correcta, pues enlista todas las posibles soluciones. A pesar de que el concepto sea sencillo de entender y aplicar, utilizar este método suele ser ineficiente, pues suelen tener una complejidad temporal demasiado alta, y requerir mayores recursos que un algoritmo que usan una técnica de diseño más adecuada para el problema. \cite{BruteForce}\\

En este experimento vamos a determinar cuál de estas dos técnicas para el diseño de algoritmos es más eficiente para resolver el problema de encontrar el par de puntos más cercanos en un arreglo ordenado.